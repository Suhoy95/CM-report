\chapter{Исследование свойств предельного цикла}

Следующим шагом в исследовании системы станет изучение свойств нашего предельного цикла
при конкретном значении параметра (возьмем $\nu = 1$): его периода (от независимой переменной) и его форме.
Данные свойства потребуются в следующих частях (\ref{lab4} и \ref{lab5}) для проверки 
нашего предельного цикла на устойчивость.

Ставя численные эксперименты, значения могут получатся точные, но все же с погрешностью.
Поэтому далее мы будем находить значение с точностью до 3-х знаков после запятой
(наше значение должно расходится не более чем на $ \epsilon = 0.5 * 10 ^{-4}$).

В программе \ref{lab3:prog:1} строится цикл методом точечных отображений Пуанкаре:
выбирается точка на оси $0_{y1}$ мы начинаем двигаться по траектории до тех пор,
пока снова не пересечет ось. При приближении к нашему предельному циклу, точки будут 
сближаться все больше и больше, поэтому будем считать траекторию предельным циклом, когда
начальная и конечная точка сблизятся по обои координатам на $\epsilon$. Периодом нашего цикла
будет количество шагов ($i + 1$) помноженных на длину шага $h$. Как видно из работы
программы, цикл имеет период $\omega = 6.663$. 

Далее можно попытаться найти аналитическую форму данного цикла, но судя по графику \ref{lab1:cycle}
форма цикла не похож на знакомые квадратичные функции и подбор аналитического вида кривой может оказаться
трудной задачей, при этом мы не сможем достигнуть такой же точности, как наше поточечное решение, полученное.
методом Эйлера. Поэтому в следующих работах будем работать с массивами $y1$ и $y2$, описывающие наш цикл.

\begin{program}
    \caption{Поиск параметров системы}
    \label{lab3:prog:1}
    \begin{verbatim}
eps = 0.5 * 10 ** -4 
y1_0, y2_0 = 0.724197, 0 # начальная точка 

y1 = [y1_0]
y2 = [y2_0]
h = 0.0001
for i in range(100000):
    # итерация метода Эйлера
    y1.append(y1[-1] + h*(y2[-1]))
    y2.append(y2[-1] + h*(-3*y2[-1] ** 3 + ny * y2[-1] - y1[-1]))
    # проверка прихода в туже точку с погрешностью
    if  np.abs(y1_0 - y1[-1]) < eps and
        np.abs(y2_0 - y2[-1]) < eps:
        # вывод результатов
        print("h={h}, i={i}, h*i={period}".format(
              h=h, i=i+1, period=h*(i+1)))
        return;
# Вывод программы 
# h=0.0001, i=66633, h*i=6.6633000000000004
    \end{verbatim}
\end{program}


