\Conclusion % заключение к отчёту
В ходе выполнения работы, была смоделирована система
обыкновенных дифференциальных уравнений(ОДУ), на примере которой
мы рассмотрели понятие придельного цикла, а
также его свойств, обсужденных на лекциях, построенных
на пособии \cite{bookdiff}.

В разделах \ref{lab4} и \ref{lab5} мы рассмотрел понятие асимптотической
орбитальной устойчивости и проверили это свойство на рассматриваемом
предельном цикле с помощью теорем Андронова-Ватта и метода
Пуанкаре соответственно.

После этого, были рассмотрено моделирование функциональных
дифференциальных уравнений (ФДУ), путем модификации
нашей системы за счет добавления разнообразных видов задержек
(постоянная, переменная, распределенная) и сравнения их влияния
на предельный цикл системы, что сильно меняло его свойства
и выходило за пределы теории ОДУ.

И последним видом уравнений, которые были промоделирваны в
этой работе, были стохастические дифференциальные уравнения,
в которых на исследуемый цикл влиял случайный шум на каждой
итерации. Дальнейший анализ таких видов уровнений можно будет
изучить в книге "i-гладкий анализ"\cite{bookananlyz}.

%%% Local Variables:
%%% mode: latex
%%% TeX-master: "rpz"
%%% End:
