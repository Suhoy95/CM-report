\chapter{Влияние распределенного запаздывания}\label{lab8}

Последним видом запаздывания, которое мы рассмотрим будет
распределенное запаздывание. Его характерной особенностью является то, что
на поведение системы так или иначе влияет вся предыстория
системы на отрезке $[t-\tau;t]$ и задается это влияние интегралом
(Как например в нашем опыте, ур. \eqref{lab8:eq1}).

\begin{equation}\label{lab8:eq1}
  \begin{cases}
      &\dot{y_1} = y_2 + \alpha * \int_{t-\tau}^t y_1^2(s)\mathrm{d}s\\
      &\dot{y_2} = -3y_2^3\ + \nu y_2 - y_1
  \end{cases}
\end{equation}

Считать интеграл мы будем методом прямоугольников, с одной
дополнительной оптимизацией: так как отрезок интегрирования
на каждом шаге меняется не сильно, а лишь сдвигается на шаг
$h$, то мы можем не пересчитывать интеграл полностью:
достаточно просто вычесть последнее слагаемое в сумме, и
добавить новое, появившееся за итерацию. Тогда вычислительная
сложность моделирования уменьшится на порядок. В остальном,
начальные значения системы и метод моделирования остается
таким же.

\myImage{($\alpha = -0.399$) Второе решение не построилось. Первое, сделав пару циклов, ушло в минус бесконечность по оси $Oy_1$}{8_y1_-0_399}{lab8:y1:1}
\myImage{($\alpha = -0.398$) Первое решение остается в пределах цикла. Второе расходится}{8_y1_-0_398}{lab8:y1:2}
\myImage{($\alpha = -0.162$) Оба решения стабилизируются вкруг цикла}{8_y1_-0_162}{lab8:y1:3}
\myImage{($\alpha = 0.07$) Небольшое положительное влияние, а второе решение уже разошлось}{8_y1_0_07}{lab8:y1:4}
\myImage{($\alpha = 0.397$) Цикл первого решения все еще строится, хоть и деформировался}{8_y1_0_397}{lab8:y1:5}
\myImage{($\alpha = 0.398$) Второе решение не построилось. Первое ушло в плюс бесконечность по оси $Oy_1$}{8_y1_0_398}{lab8:y1:6}

\clearpage
В данном эксперименте можно увидеть одну необычную закономерность:
относительно параметра $\alpha$ первое решение ведет себя
очень похоже, что при $-0.398$, что при $0.398$. За исключением того,
что петелька зеркально переворачивается и начинает расходиться в
противоположное направление.

Также можно увидеть, что у в этом наборе опытов распределенное
запаздывание быстрее всех повлияло на предельный цикл:
при постоянном запаздывании решение расходилось при $\alpha > 1.24$,
при переменном $\alpha > 0.691$, при распределенном же
разошлось при $\alpha > 0.398$.

Но стоит заметить, что этот
вывод характерен только для этой серии экспериментов:
при изменении системы или формы задержки все может поменяться
непредсказуемым образом. Как пример этого, можно сравнить эти результаты
с экспериментами в приложении \ref{app}: там влияние переменной задержки
при отрицательных $\alpha$ не разрушает цикл системы очень долго.
