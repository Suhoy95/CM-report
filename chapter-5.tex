\chapter{Определение устойчивости с помощью метода Пуанкаре}\label{lab5}

В этом разделе, рассмотрим другой метод определения асимптотической орбитальной
устойчивости циклического решения, предложенный Анри Пуанкаре. Он определен
только для двумерных автономных систем, что нам подходит.

\begin{theorem}
    Теорема Пуанкаре.
    периодическое решение системы будет асимптотически
    орбитально устойчивым тогда и только тогда, когда интеграл дивергенции вдоль
    решения будет меньше нуля.
\end{theorem}

Таким образом, нам остается лишь взять вычисленные в уравнении \ref{lab4:eq:2}
производные, составляющие дивергенцию системы. После чего вычислить интеграл
(возьмем метод прямоугольников). Программа \ref{lab5:prog:1} производит данное
вычисление, и результат подтверждает полученный в предыдущей работе результат:
предельный цикл удовлетворяет критерию Пуанкаре, значит асимптотически орбитально
устойчив.

\begin{program}
    \caption{Вычисление интеграла от дивергенции системы}
    \label{lab5:prog:1}
    \begin{verbatim}
def integral_from_div(cycle_y1, cycle_y2):
    """
    Вычисление интеграла от дивергенции системы
    вдоль цикла
    """
    sum = 0
    for j in range(len(cycle_y1)):
        sum += -9 * cycle_y2[j] ** 2 + ny
    sum *= h
    return sum
# Основная программа
cycle_y1, cycle_y2 = line(0.724197, 0)
s = integral_from_div(cycle_y1, cycle_y2)
print("Integral of the divergence: {}".format(s))
# Вывод программы:
# Integral of the divergence: -7.059887304427718
    \end{verbatim}
\end{program}