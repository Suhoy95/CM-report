%% Преамбула TeX-файла

% 1. Стиль и язык
\documentclass[utf8x]{G7-32} % Стиль (по умолчанию будет 14pt)
\usepackage[T2A]{fontenc}
\usepackage[russian]{babel}
% Остальные стандартные настройки убраны в preamble.inc.tex.
\include{preamble.inc}

% Настройки листингов.
\include{listings.inc}

% Полезные макросы листингов.
% Любимые команды
\newcommand{\Code}[1]{\textbf{#1}}

\newcommand{\myImage}[3]{
\begin{figure}[!ht]
    \centering
    \includegraphics[width=\textwidth]{figures/#2}
    \caption{#1}
    \label{#3}
\end{figure}
}


% Пакет для листингов кода
\usepackage{verbatim}
\usepackage{float}
\floatstyle{ruled}
\newfloat{program}{th}{lop}
\floatname{program}{Листинг}


\newtheorem{theorem}{Теорема}
\newtheorem{definition}{Определение}

\begin{document}

\pagestyle{empty}
\begin{center}
  Министерство образования и науки Российской Федерации\\
  ФГАОУ ВПО  «УрФУ имени первого Президента России Б. Н. Ельцина»\\
  Институт радиоэлектроники и информационных технологий - РтФ\\
  Департамент информационных технологий и автоматики
  \par
  \vspace{4.5cm}
  \Large{
    Исследование предельных циклов нелиненой системы
    
    \par
    \vspace{0.5cm}

    ОТЧЕТ\\
    по лабораторной работе
  }

  \vspace{4cm}
  {
    Преподаватель: \hfill Пименов Владимир Германович
  }
  \par
  {
    Студент: \hfill Сухоплюев Илья Владимирович
  }
  \par
  {
    Группа: \hfill РИ-440001
  }

  \par
  \vspace{3.5cm}
  Екатеринбург\\
  2017
\end{center}
  

\frontmatter % выключает нумерацию ВСЕГО; здесь начинаются ненумерованные главы: реферат, введение, глоссарий, сокращения и прочее.

% Команды \breakingbeforechapters и \nonbreakingbeforechapters
% управляют разрывом страницы перед главами.
% По-умолчанию страница разрывается.

% \nobreakingbeforechapters
% \breakingbeforechapters

% Также можно использовать \Referat, как в оригинале
\begin{abstract}

Работа описывает исследование параметризованной нелинейной системы
на наличие предельных циклов. В ходе исследования системы,
рассматриваются следующие вопросы: нахождение параметра системы, при котором 
наблюдается предельный цикл; поиск параметра, где наблюдается бифуркация системы;
исследование свойств обнаруженного предельного цикла и определение характера его
устойчивости. Данные задачи изучаются путем проведения численных экспириментов
с помощью интерпритатора Python 3.5 и математических библиотек (numpy, matplotlib).

\end{abstract}

%%% Local Variables: 
%%% mode: latex
%%% TeX-master: "rpz"
%%% End: 


\pagestyle{plain}

\tableofcontents

\mainmatter % это включает нумерацию глав и секций в документе ниже

\chapter{Поиск предельного цикла}

Рассмотрим исследуемую систему (уравнение \ref{lab:eq:1},
$\nu$ - параметр системы). Она описывается
уравнением от одной фазовой переменной $x$. Уравнение дифференциальное, второго
порядка и, в силу слагаемого $3\dot{x}^3$, нелинейное. Решение такого уравнения
аналитическими методами является довольно сложной задачей, поэтому нашим
методом исследования будет построение численных экспериментов, описывающих
данную систему при определенном параметре $\nu$.

\begin{equation}\label{lab:eq:1}
  \ddot{x} + 3 \dot{x}^3 - \nu\dot{x} + x = 0
\end{equation}

Однако, в таком виде уравнение \ref{lab:eq:1} не является удобным для
моделирования. Поэтому приведем его к канонической форме от двух переменных, с
помощью замены \ref{lab:eq:2}, получив уравнение от двух фазовых переменных
$y_1$ и $y_2$ (Система уравнений \ref{lab:eq:3}). В дальнейшем, мы будем
пользоваться описанием нашей системы именно в таком виде.

\begin{equation}\label{lab:eq:2}
  \begin{cases}
    &y_1 = x \\
    &y_2 = \dot{x}
  \end{cases}
\end{equation}

\begin{equation}\label{lab:eq:3}
  \begin{cases}
    &\dot{y_1} = y_2 \\
    &\dot{y_2} = -3y_2^3\ + \nu y_2 - y_1
  \end{cases}
\end{equation}

Преобразовав систему к удобному для нас виду, перейдем к первой части работы --
нахождения такого параметра $\nu$, при котором наблюдается предельный цикл.

Для начала, дадим определение искомому объекту.

\begin{definition}\label{lab:def:cycle}
  Предельным циклом будем называть замкнутую изолированную траекторию
  в фазовом пространстве, подразумевая замкнутость в смысле периодичности
  поведения системы.
\end{definition}

Таким образом, нам нужно построить фазовый портрет нашей системы, на котором
нужно будет обнаружить искомую замкнутую линию. Для этого, зная зависимость
значения производных от их координат, можно с помощью функции
\textit{streamplot}\cite{streamplot} построить фазовый портрет (Программа
\ref{lab1:prog:1}, в качестве параметра для начала возьмем $\nu = 1$).

\begin{program}
  \caption{Построение фазового портрета}
  \label{lab1:prog:1}
  \begin{verbatim}
# Подключение используемых библиотек
# В дальнейшем является постоянным и опускается в листингах
# Полный исходный код программы можно найти в приложении
import matplotlib.pyplot as plt
import numpy as np

# Параметр системы
nu = 1

# создание сетки 100х100 точек в области [-3;3]x[-3;3]
Y, X = np.mgrid[-3:3:100j, -3:3:100j]

# вычисление фазовых векторов на сетке
Y1 = Y
Y2 = -3 * Y ** 3 + nu * Y - X

# построение фазового портрета
fig0, ax0 = plt.subplots()
plt.streamplot(X, Y, Y1, Y2)

# показать построенные графики (опускается в дальнейшем)
plt.show()
  \end{verbatim}
\end{program}
\clearpage


\begin{figure}[thp]
  \centering
  \includegraphics[width=\textwidth]{figures/1_streamplot}
  \caption{Поиск предельного цикла построением фазового портрета}
  \label{lab1:streamplot}
\end{figure}

На графике \ref{lab1:streamplot} изображен результат работы нашей программы.
В данном случае значение параметра оказалось оптимальным: можно видеть, как
изоклины сходятся к наклоненному прямоугольнику в центре графика.

Теперь, чтобы убедится наверняка, что траектории сходятся вокруг этого цикла
и там нет разрывов, построим две линии методом Эйлера снаружи и внутри наблюдаемого
цикла (Программа \ref{lab1:prog:2}).

\begin{program}
  \caption{Использование метода Эйлера для проверки предельного цикла}
  \label{lab1:prog:2}
  \begin{verbatim}
# функция построение кривой методом Эйлера
def line(y1_0, y2_0):
    y1 = [y1_0]
    y2 = [y2_0]
    h = 0.01 # длина шага
    for i in range(2000): # 2000 - количество итераций
        y1.append(y1[i] + h*(y2[i]))
        y2.append(y2[i] + h*(-3*y2[i] ** 3 + nu*y2[i] - y1[i]))
    # отображение кривой на графике
    ax0.plot(y1, y2)

# построение двух кривых, начинающихся внутри и
# вне предполагаемого предельного цикла
line(0.1, 0.1)
line(2, 2)
  \end{verbatim}
\end{program}

\clearpage

\begin{figure}[thp]
  \centering
  \includegraphics[width=\textwidth]{figures/1_cycle}
  \caption{Обнаружение аттрактора методом Эйлера}
  \label{lab1:cycle}
\end{figure}

На рисунке \ref{lab1:cycle} мы можем видеть две линии, начинающиеся из точек
$(0.1, 0.1)$ и $(2, 2)$. Эти линии сходятся сближаются к искомому предельному
циклу системы.

\chapter{Поиск точек бифуркаций}

Найдя предельный цикл в системе \eqref{lab:eq:2}, мы можем перейти к следующему
этапу нашего исследования -- определения всех значений параметра, при которых
наблюдается данный цикл.

В силу того, что наша система рассматривается дифференциальным уравнением,
поведение системы будет меняться плавно на промежутках, разделенных так
называемыми, точками \textit{бифуркации}(точки, в которых происходит изменение
поведения системы).

\begin{definition}
    Точка бифуркации - значение параметра системы, при котором наблюдается
    качественное изменение поведения системы.
\end{definition}

Чтобы нам было удобно наблюдать изменение системы от параметра без перезапуска
программы, мы обернем построения в функцию и добавим в нашу программу слайдер --
бегунок, которым можно будет менять значение параметра $\nu$.

\begin{figure}
    \centering
    \includegraphics[width=0.8\textwidth]{figures/2_point_-2_2}
    \caption{Стационарная точка системы при $\nu = -2.2$}
    \label{lab2:point_-2}
\end{figure}

Начиная с отрицательных значений (от -10) мы наблюдаем сильное стремление
к центру координат -- стационарной точки системы (График \ref{lab2:point_-2}).

\begin{figure}[!ht]
    \centering
    \includegraphics[width=0.8\textwidth]{figures/2_point_-0_1}
    \caption{Стационарная точка системы при $\nu = -0.1$}
    \label{lab2:point_0}
\end{figure}

При приближении параметра к нулю, поведение системы искривляется в овальную
форму, но линии медленно сходятся к нулю (График \ref{lab2:point_0}).

\begin{figure}[!ht]
    \centering
    \includegraphics[width=0.8\textwidth]{figures/2_cycle_0_1}
    \caption{Появление цикла при $\nu = 0.1$}
    \label{lab2:cycle_0_1}
\end{figure}

Как только мы переступаем нулевое значение параметра, наши траектории
останавливаются значительно раньше -- мы начинаем наблюдать знакомый нам
предельный цикл, но в меньших размерах (График \ref{lab2:cycle_0_1}).

\begin{figure}[!ht]
    \centering
    \includegraphics[width=0.8\textwidth]{figures/2_cycle_20}
    \caption{Расширение предельного цикла при увеличении параметра ($\nu = 20$)}
    \label{lab2:cycle_20}
\end{figure}

Увеличивая $\nu$ дальше, остается наблюдать за ростом цикла (График
\ref{lab2:cycle_20}).

Из полученных наблюдений можно выдвинуть гипотезу: на отрицательной полуоси
исследуемая система сходится в стационарную точку; в положительной же оси
наблюдается предельный цикл, который увеличивается в зависимости от параметра
системы.

Стоит отметить, что наличие или отсутствие предельного цикла на границе
($\nu = 0$) мы выявить не можем, так как при приближении к параметра к нулю,
чтобы быть уверенным в наличии стационарной точки или цикла, приходится
увеличивать точность вычислений. В конце концов, когда точность увеличить
не удается, нам остается только предполагать: толи линии сошлись к циклу, толи
они не достигли нуля из-за недостаточного кол-ва шагов в методе Эйлера.

С учетом этого замечания, можно выдвинуть еще одну гипотезу: так как изменение
поведения в системе происходит настолько плавно, что нам не удается уловить
момент, когда мы наблюдаем стационарную точку, а когда предельный цикл.
То есть мы можем говорить, что наблюдается \textit{мягкая бифуркация системы}.

\chapter{Исследование свойств предельного цикла}

Следующим шагом в исследовании системы станет изучение свойств нашего
предельного цикла при конкретном значении параметра (возьмем $\nu = 1$):
нахождение его периода (от независимой переменной) и его форму. Данные свойства
потребуются в следующих частях (\ref{lab4} и \ref{lab5}) для проверки
характера его устойчивости.

Ставя численные эксперименты, значения могут получатся точные, но все же с
погрешностью. Поэтому далее мы будем находить значение с точностью до 3-х знаков
после запятой (т. е. наше значение должно расходится не более чем на
$\epsilon = 0.5 * 10 ^{-4}$).

В программе \ref{lab3:prog:1} строится цикл методом точечных отображений Пуанкаре:
выбирается точка на оси $0_{y1}$, от которой мы начинаем двигаться по траектории
до тех пор, пока снова не пересечет ось. При приближении к нашему предельному
циклу, точки будут сближаться все больше и больше. Таким образом будем считать
траекторию предельным циклом, когда начальная и конечная точка сблизятся по
обоим координатам ближе чем на $\epsilon$. Периодом нашего цикла будет
количество затраченных шагов ($i + 1$) помноженных на длину шага $h$.
Как видно из работы программы, цикл имеет период $\omega = 6.663$.

Далее можно попытаться найти аналитическую форму данного цикла, но судя по
графику \ref{lab1:cycle} форма цикла не похож на знакомые квадратичные функции и
подбор аналитического вида кривой может оказаться трудной задачей, при этом мы
не сможем достигнуть такой же точности, как наше поточечное решение, полученное
методом Эйлера. Поэтому в следующих работах будем работать с массивами $y1$ и
$y2$, описывающие наш цикл.

\begin{program}
    \caption{Поиск параметров системы}
    \label{lab3:prog:1}
    \begin{verbatim}
eps = 0.5 * 10 ** -4
y1_0, y2_0 = 0.724197, 0 # начальная точка

y1 = [y1_0]
y2 = [y2_0]
h = 0.0001
for i in range(100000):
    # итерация метода Эйлера
    y1.append(y1[-1] + h*(y2[-1]))
    y2.append(y2[-1] + h*(-3*y2[-1] ** 3 + ny * y2[-1] - y1[-1]))
    # проверка прихода в туже точку с погрешностью
    if  np.abs(y1_0 - y1[-1]) < eps and
        np.abs(y2_0 - y2[-1]) < eps:
        # вывод результатов
        print("h={h}, i={i}, h*i={period}".format(
              h=h, i=i+1, period=h*(i+1)))
        return;
# Вывод программы
# h=0.0001, i=66633, h*i=6.6633000000000004
    \end{verbatim}
\end{program}



\chapter{Определение устойчивости через мультипликаторы системы}\label{lab4}
\chapter{Определение устойчивости с помощью метода Пуанкаре}\label{lab5}

В этом разделе, рассмотрим другой метод определения асимптотической орбитальной
устойчивости циклического решения, предложенный Анри Пуанкаре. Он определен
только для двумерных автономных систем, что нам подходит.

\begin{theorem}
    Теорема Пуанкаре.
    периодическое решение системы будет асимптотически
    орбитально устойчивым тогда и только тогда, когда интеграл дивергенции вдоль
    решения будет меньше нуля.
\end{theorem}

Таким образом, нам остается лишь взять вычисленные в уравнении \ref{lab4:eq:2}
производные, составляющие дивергенцию системы. После чего вычислить интеграл
(возьмем метод прямоугольников). Программа \ref{lab5:prog:1} производит данное
вычисление, и результат подтверждает полученный в предыдущей работе результат:
предельный цикл удовлетворяет критерию Пуанкаре, значит асимптотически орбитально
устойчив.

\begin{program}
    \caption{Вычисление интеграла от дивергенции системы}
    \label{lab5:prog:1}
    \begin{verbatim}
def integral_from_div(cycle_y1, cycle_y2):
    """
    Вычисление интеграла от дивергенции системы
    вдоль цикла
    """
    sum = 0
    for j in range(len(cycle_y1)):
        sum += -9 * cycle_y2[j] ** 2 + ny
    sum *= h
    return sum
# Основная программа
cycle_y1, cycle_y2 = line(0.724197, 0)
s = integral_from_div(cycle_y1, cycle_y2)
print("Integral of the divergence: {}".format(s))
# Вывод программы:
# Integral of the divergence: -7.059887304427718
    \end{verbatim}
\end{program}
\chapter{Влияние постоянного запаздывания}\label{lab6}

С этой главы мы познакомимся с \textit{функциональными дифференциальными
уравнениями (ФДУ)}, а точнее с системами в которых присутствует элемент
запаздывания и как он влияет на поведение исследуемой системы.

\begin{definition}
    Функциональным дифференциальным уравнением с постоянным запаздыванием,
    мы будем называть уравнение вида \ref{lab6:eq1}.
\end{definition}

\begin{equation}\label{lab6:eq1}
    \begin{cases}
        &\dot{x} = f(t, x, x_t(t - \tau)) \\
        &x(0) = x_0 \\
        &x_{t_0}(\cdot) = \begin{Bmatrix}y_0(t); t \in [\tau;0)\end{Bmatrix}
    \end{cases}
\end{equation}

Посмотрев на него, можно выделить основные особенности от обычных дифференциальных
уравнений, помимо зависимости от $t$ и $x$, у нас появляется зависимость от
значения фазовой переменной в момент $x_t(t-\tau)$ (Здесь и далее, $x_t$ - \textit{функция
предыстории}, описывающая значения фазовой переменной на интервале $[t-\tau;t)$).
Так же, чтобы получить однозначное решение системы \ref{lab6:eq1}, понадобилось
задать предысторию до момента $t_0 = 0$ (третье равенство системы).

\section{Метод Эйлера с кусочно постоянной интерполяцией}
% 2. Не описаны численные методы, их можно извлечь лишь косвенно из кодов программ.
Для того, чтобы промоделировать такую систему нам потребуется
модернизировать наш метод Эйлера. Суть модификации заключается
в том, как мы будем считать элемент $x_t(t - \tau)$ в нашей
дискретной схеме подсчета.

\begin{definition}
    Интерполяцией $u_t$, назовем дискретную функцию определенную на отрезке
    $[t-\tau, t]$, которая описывает дискретную историю модели в момент времени
    $t$.
\end{definition}

Таким образом, интерполяция - это дискретное приближение
функции предыстории $x_t$. Понятно, что интерполяцию можно
построить разными способами, и более того, это может влиять
на сходимость нашего метода. В нашем случае, будет достаточно
простой кусочно-линейной интерполяции (Уравнение \ref{lab6:eq2}):

\begin{equation}\label{lab6:eq2}
    u(t) = \begin{cases}
        &u_{i-1}, t \in [t_{i-1}, t_i) \\
        &y(t), t \leq t_0
    \end{cases}
\end{equation}

Соответственно представив в метод Эйлера вместо $x_t(t - \tau)$
определенную интерполяцию, мы получим его модификацию, называемую
\textit{методом Эйлера с кусочно постоянной интерполяцией}.

\begin{theorem}
    Метод Эйлера с кусочно постоянной интерполяцией имеет порядок
    сходимости $p = 1$.
\end{theorem}


% 1. Начиная с раздела 6.1 встречаются конкретные эксперименты с системами с запаздыванием,
% поэтому нужно указывать начальные условия, в том числе и функциональные.

\section{Численный эксперимент}

Проведем первый эксперимент с постоянным запаздыванием.
Для этого, введем в нашу систему слагаемое
$\alpha * y_1(t-\tau)$ (Ур. \ref{lab6:eq2}). При параметре
$\alpha = 0$ оно не влияет на систему, следовательно мы будем
наблюдать наш предельный цикл. Постепенно изменяя этот параметр,
мы будем наблюдать на изменения характера системы.

\begin{equation}\label{lab6:eq3}
\begin{cases}
    &\dot{y_1} = y_2 + \alpha * y_1(t-\tau)\\
    &\dot{y_2} = -3y_2^3\ + \nu y_2 - y_1
\end{cases}
\end{equation}

В прошлых экспериментах мы рассматривали два решения: от точки
$(0.1, 0.1)$ и от точки $(2, 2)$. Теперь же нам еще нужна
предыстория нашей системы, поэтому мы просто возьмем ее константной
(ур. \ref{lab6:eq4} и \ref{lab6:eq5}).

\begin{equation}\label{lab6:eq4}
  \begin{cases}
    &y_{1_{t_0}} = 0.1, t \in [-\tau, 0]\\
    &y_{2_{t_0}}} = 0.1, , t \in [-\tau, 0]
  \end{cases}
\end{equation}

\begin{equation}\label{lab6:eq5}
  \begin{cases}
    &y_{1_{t_0}} = 2, t \in [-\tau, 0]\\
    &y_{2_{t_0}} = 2, , t \in [-\tau, 0]
  \end{cases}
\end{equation}

Стоит отметить, модернизацию нашей программы в техническом
виде, с помощью слайдера оказалось неудобным отлаживать
подбор необходимого параметра. Поэтому, чтобы упростить
эксперименты и не перезапускать программу, вместо слайдера
был поставлен \textmd{TextBox} - поле для ввода текста, куда мы
можем ввести любое, необходимое нам, значение (пересчет
графика произойдет при нажатии на клавишу \textmd{Enter}).
При желании, с подробностями использования этого виджета
можно ознакомиться в приложении.

\myImage{($\alpha = -0.65$) Решение от точки (0.1, 0.1) хаотично расходится от центра.
Решение от точки (2, 2) не смогло построится}{6_y1_-0_65}{lab6:y1:1}
\myImage{($\alpha = -0.649$) Оба решения построились. Отчетливо наблюдается жесткость
системы, цикл сохранился, но об изолированности говорить не приходится}{6_y1_-0_649}{lab6:y1:2}
\myImage{($\alpha = 0.426$) Решение все еще сходится к циклу, но постепенно разбалтывается}{6_y1_0_426}{lab6:y1:3}
\myImage{($\alpha = 0.427$) Второе решение перестало моделироваться. Первое все еще сходится к циклу}{6_y1_0_427}{lab6:y1:4}
\myImage{($\alpha = 1$) У первого решения появляются две петельки и траектория становится более хаотична}{6_y1_1_0}{lab6:y1:5}
\clearpage
\myImage{($\alpha = 1.24$) Первое решение хаотично изменятся в пределах цикла}{6_y1_1_14}{lab6:y1:6}

Нам не удавалось сильно изменить параметр $\alpha$:
сразу же начинали расходиться решения (или решение);
проявлялось, своего рода, хаотичное поведение; исчезали циклы.
То есть мы увидели, как сильно влияет на поведение системы
даже самая простая, в плане понимания, задержка.

При добавлении подобного слагаемого во второе уравнение можно
пронаблюдать другие бифуркации. При желании, эти эксперименты
можно посмотреть в Приложении \ref{app:const}.

\chapter{Влияние переменного запаздывания}

Следующим видом запаздывания, которое мы рассмотрим, будет
переменное. Оно отличается тем, что значение истории,
которое мы будем смотреть в ра

\begin{equation}\label{lab6:eq3}
  \begin{cases}
      &\dot{y_1} = y_2 + \alpha * y_1(t-\tau*\sin(\frac{2t}{T}))\\
      &\dot{y_2} = -3y_2^3\ + \nu y_2 - y_1
  \end{cases}
\end{equation}

\myImage{($\alpha = -0.727$)}{7_y1_-0_727}{lab7:y1:1}
\myImage{($\alpha = -0.42$)}{7_y1_-0_42}{lab7:y1:2}
\myImage{($\alpha = 0.38$)}{7_y1_0_38}{lab7:y1:3}
\myImage{($\alpha = 0.379$)}{7_y1_0_379}{lab7:y1:4}
\myImage{($\alpha = 0.691$)}{7_y1_0_691}{lab7:y1:5}

\chapter{Влияние распределенного запаздывания}\label{lab8}

Последним видом запаздывания, которое мы рассмотрим будет
распределенное запаздывание. Его характерной особенностью является то, что
на поведение системы так или иначе влияет вся предыстория
системы на отрезке $[t-\tau;t]$ и задается это влияние интегралом
(Как например в нашем опыте, ур. \ref{lab8:eq1}).

\begin{equation}\label{lab8:eq1}
  \begin{cases}
      &\dot{y_1} = y_2 + \alpha * \int_{t-\tau}^t y_1^2(s)\mathrm{d}s\\
      &\dot{y_2} = -3y_2^3\ + \nu y_2 - y_1
  \end{cases}
\end{equation}

Считать интеграл мы будем методом прямоугольников, с одной
дополнительной оптимизацией: так как отрезок интегрирования
на каждом шаге меняется не сильно, а лишь сдвигается на шаг
$h$, то мы можем не пересчитывать интеграл полностью:
достаточно просто вычесть последнее слагаемое в сумме, и
добавить новое, появившееся за итерацию. Тогда вычислительная
сложность моделирования уменьшится на порядок. В остальном,
начальные значения системы и метод моделирования остается
таким же.

\myImage{($\alpha = -0.399$) Второе решение не построилось. Первое, сделав пару циклов, ушло в минус бесконечность по оси $Oy_1$}{8_y1_-0_399}{lab8:y1:1}
\myImage{($\alpha = -0.398$) Первое решение остается в пределах цикла. Второе расходится}{8_y1_-0_398}{lab8:y1:2}
\myImage{($\alpha = -0.162$) Оба решения стабилизируются вкруг цикла}{8_y1_-0_162}{lab8:y1:3}
\myImage{($\alpha = 0.07$) Небольшое положительное влияние, а второе решение уже разошлось}{8_y1_0_07}{lab8:y1:4}
\myImage{($\alpha = 0.397$) Цикл первого решения все еще строится, хоть и деформировался}{8_y1_0_397}{lab8:y1:5}
\myImage{($\alpha = 0.398$) Второе решение не построилось. Первое ушло в плюс бесконечность по оси $Oy_1$}{8_y1_0_398}{lab8:y1:6}

\clearpage
В данном эксперименте можно увидеть одну необычную закономерность:
относительно параметра $\alpha$ первое решение ведет себя
очень похоже, что при $-0.398$, что при $0.398$. За исключением того,
что петелька зеркально переворачивается и начинает расходиться в
противоположное направление.

Также можно увидеть, что у в этом наборе опытов распределенное
запаздывание быстрее всех повлияло на предельный цикл:
при постоянном запаздывании решение расходилось при $\alpha > 1.24$,
при переменном $\alpha > 0.691$, при распределенном же
разошлось при $\alpha > 0.398$.

Но стоит заметить, что этот
вывод характерен только для этой серии экспериментов:
при изменении системы или формы задержки все может поменяться
непредсказуемым образом. Как пример этого, можно сравнить эти результаты
с экспериментами в приложении \ref{app}: там влияние переменной задержки
при отрицательных $\alpha$ не разрушает цикл системы очень долго.

\chapter{Влияние случайного шума}\label{lab9}
Последним видом моделируемых систем, с которым мы познакомимся
в этой работе, будет моделирование \textit{стохастического
дифференциального уравнения}. Оно описывает систему, на работу
которой в каждый момент времени может оказывать влиять
случайное искажение (шум). Обычно, в качестве шума рассматривают
реализацию выборки стандартного нормального распределения,
поэтому в наших экспериментах мы будем использовать его.

В том виде, в котором описывались предыдущие дифференциальные
уравнения, мы описывать данное уравнение не можем, однако
такое уравнение хорошо описывается через интегральную форму
(ур. \eqref{lab9:eq:integral}).

\begin{equation}\label{lab9:eq:integral}
    x(t) = x_0 + \int_{t_0}^t f(s,x(s))\mathrm{d}s +
                 \int_{t_0}^t \sigma(W, x(W))\mathrm{d}W
\end{equation}

Где второе слагаемое интеграл по Винеровскому процессу,
как раз и описывает приращение ошибки (шума) в системе.

% \begin{definition}
%     Винеровский (Броуновский) случайный процесс $W(t)$ определяется
%     следующими аксиомами:
%     \begin{itemize}
%         \item $\forall t: W(t) ~ N(a = 0, \sigma)$
%         \item $\forall s,t: s \noteq t, W(t), W(s)$ - независимые случайные величины.
%         \item
%     \end{itemize}
% \end{definition}

От такого вида уравнения можно перейти к СДУ в виде
дифференциалов (ур. \eqref{lab9:eq:dif}).

\begin{equation}\label{lab9:eq:dif}
    \mathrm{d}x = f(t, x)\mathrm{d}t +
                  \sigma(t, x(t))\mathrm{d}W
\end{equation}

Для нашей системы, такое уравнение примет вид \eqref{lab9:eq:our}.

\begin{equation}\label{lab9:eq:our}
\begin{cases}
    &\mathrm{d}y_1 = y_2\mathrm{d}t + \sigma * \mathrm{d}W\\
    &\mathrm{d}y_2 = (-3y_2^3\ + \nu y_2 - y_1)\mathrm{d}t
\end{cases}
\end{equation}

В этом уравнении мы добавляем шум к изменению первой координаты
системы. Тут Виноровский процесс описывается через стандартное
нормальное распределение, а за счет коэффициента $\sigma$ мы
сможем влиять на среднеквадратичное отклонение данного процесса,
а следовательно и влияние данного процесса на систему.

Для моделирования такой системы используется другая
модификация метода Эйлера - \textit{метод Эйлера-Марайамы}.
Который для нашей системы будет выглядеть так:

\begin{equation}\label{lab9:eq:method}
\begin{cases}
    &y_1^{i+1} = y_1^i + h y_2^i + \sigma * \sqrt{h} W^i\\
    &y_2^{i+1} = y_2^i + h (-3(y_2^i)^3\ + \nu y_2^i - y_1^i)
\end{cases}
\end{equation}

Где $W^i$ - реализация генератора случайных чисел стандартного
нормального распределения. Начальные значения в текущем
эксперименте не принципиальны: в нашем случа берутся теже точки
$(0.1,0.1)$ и $(2,2)$, к которым добавляется случайное воздействие
$\sigma W^0$.

\begin{definition}
    Метод моделирующий стохастическое уравнение сходится с
    порядком $p$ в сильном смысле, если
    $\exists C : M(|X(T) - x^N|) \leq Ch^p$, где $X(t)$ - настоящее решение
    данного уравнения, а ${x}_i$ - решение, полученное с помощью метода.
\end{definition}

\begin{definition}
    Метод моделирующий стохастическое уравнение сходится с
    порядком $p$ слабо, если
    $\exists C : M(\int_{t_0}^T(X(t) - x^i)^2dt)^(\frac{1}{2}) \leq Ch^p$
\end{definition}

\begin{theorem}
    Метод Эйлера-Марайамы сходится, как в сильном, так и в слабом
    смысле с порядком $p = \frac{1}{2}$.
\end{theorem}

\myImage{($\sigma = 0.1$) видно, что цикл сохранил форму, но контур стал размываться вдоль оси $Oy_1$}{9_y1_0_1}{lab9:y1:1}
\myImage{($\sigma = 3$) Цикл напоминает размытое пятно, но значения решения не рассходятся}{9_y1_3_ok}{lab9:y1:2}
\myImage{($\sigma = 3$) Значение $\sigma$ не поменялось, но при определенной выборке решение разошлось}{9_y1_3}{lab9:y1:3}

\clearpage
Таким образом видно, что при добавлении шума в небольшом количестве,
поведение системы остается в пределах известного поведения, хоть
и становится слишком размытым. При сильном увеличении влияния
процесс может, как разойтись, так и нет, в зависимости от
реализации выборки случайной величины.



\backmatter %% Здесь заканчивается нумерованная часть документа и начинаются ссылки и
            %% заключение

\Conclusion % заключение к отчёту
Заключение будет, когда все закончится\cite{bookdiff}.\cite{bookananlyz}

%%% Local Variables:
%%% mode: latex
%%% TeX-master: "rpz"
%%% End:


\include{81-biblio}

\appendix   % Тут идут приложения

\chapter{Исходный код программ}
\label{appendix:src}

\section{Поиск предельного цикла}
\verbatiminput{listings/lab1.py}
\clearpage

\section{Исследование точек бифуркации системы}
\verbatiminput{listings/lab2.py}
\clearpage

\section{Исследование параметров найденного предельного цикла}
\verbatiminput{listings/lab3.py}
\clearpage

\section{Проверка устойчивости теоремой о мультипликаторах}
\verbatiminput{listings/lab4.py}
\clearpage

\section{Проверка устойчивости цикла методом Пуанкаре}
\verbatiminput{listings/lab5.py}
\clearpage

\section{Влияние постоянного запаздывания}
\verbatiminput{listings/lab6.py}
\clearpage

\section{Влияние переменного запаздывания}
\verbatiminput{listings/lab7.py}
\clearpage

\section{Влияние расспределенного запаздывания}
\verbatiminput{listings/lab8.py}
\clearpage

\section{Влияние случайного шума}
\verbatiminput{listings/lab9.py}
\clearpage

%%% Local Variables:
%%% mode: latex
%%% TeX-master: "rpz"
%%% End:

\chapter{Дополнительные эксперименты}\label{app}

В данном приложении собраны численные эксперименты, которые
не вошли в основное содержание отчета, чтобы не загромождать
изложение материала, но тем не менее не стали менее важными
в контексте проведенных лабораторных работ.

\section{Влияние постоянного запаздывания на вторую
переменную системы}\label{app:const}

В данном эксперименте мы добавляем постоянное запаздывание
ко второй переменной системы, получая уравнение \eqref{app:const:1}.
Начальные значения решений и метод аналогичны разделу \ref{lab6}.

\begin{equation}\label{app:const:1}
    \begin{cases}
        &\dot{y_1} = y_2\\
        &\dot{y_2} = -3y_2^3\ + \nu y_2 - y_1  + \alpha * y_1(t-\tau)
    \end{cases}
\end{equation}

\myImage{($\alpha = -5.1$) При отрицательном параметре изменяется форма цикла. Можно предполагать,
что предельность данного цикла остается}{6_y2_-5_1}{lab6:y2:1}
\myImage{($\alpha = 0.5$) Цикл начинает раскручиваться в правую сторону}{6_y2_0_5}{lab6:y2:2}
\myImage{($\alpha = 0.99$) Наблюдается очень плотная спираль}{6_y2_0_99}{lab6:y2:3}
\myImage{($\alpha = 1$) Потенциальная точка бифуркации. Два решения сошлись к двум разным циклам}{6_y2_1_0}{lab6:y2:4}
\myImage{($\alpha = 1.02$) Даже небольшое увеличение $\alpha$ изменило систему}{6_y2_1_02}{lab6:y2:5}
\clearpage
\myImage{($\alpha = 1.025$) Решение начинает увеличиваться по $y_1$}{6_y2_1_025}{lab6:y2:6}
\myImage{($\alpha = 1.13$) В интервале от 600 до 1000 наблюдаются резкие колебания}{6_y2_1_13}{lab6:y2:7}

\clearpage
\section{Влияние переменного запаздывания на вторую
переменную системы}\label{app:changeable}

В данном эксперименте мы добавляем переменное запаздывание
ко второй переменной системы, получая уравнение \eqref{app:change:1}.
Начальные значения решений и метод аналогичны разделу \ref{lab7}.

\begin{equation}\label{app:change:1}
    \begin{cases}
        &\dot{y_1} = y_2\\
        &\dot{y_2} = -3y_2^3\ + \nu y_2 - y_1 + \alpha * y_1(t-\tau*\sin(\frac{2t}{T}))
    \end{cases}
\end{equation}

\myImage{($\alpha = -21$) В цикле начались резкие колебания}{7_y2_-21}{lab7:y2:1}
\myImage{($\alpha = -11$) Зарождения колебаний в цикле }{7_y2_-11}{lab7:y2:2}
\myImage{($\alpha = 0.7$) Цикл меняет свою форму и напоминает виниловую пластинку}{7_y2_0_7}{lab7:y2:3}
\myImage{($\alpha = 0.99$) Цикл начинает смещаться вправо }{7_y2_0_99}{lab7:y2:4}
\myImage{($\alpha = 1$) Жесткая бифуркация: каждое решение сошлось к своему циклу}{7_y2_1}{lab7:y2:5}
\myImage{($\alpha = 1.1$) Оба решения начали уходить в бесконечность вдоль асимптоты}{7_y2_1_1}{lab7:y2:6}

\clearpage
\section{Влияние распределенного запаздывание на вторую
переменную системы}

В данном эксперименте мы добавляем распреденое запаздывание
ко второй переменной системы, получая уравнение \eqref{app:raspr:1}.
Начальные значения решений и метод аналогичны разделу \ref{lab8}.

\begin{equation}\label{app:raspr:1}
  \begin{cases}
      &\dot{y_1} = y_2\\
      &\dot{y_2} = -3y_2^3\ + \nu y_2 - y_1 + \alpha * \int_{t-\tau}^t y_1^2(s)\mathrm{d}s
  \end{cases}
\end{equation}

\myImage{($\alpha = -0.45$) Хаотично покрутившись в районе цикла, первое решение разошлось влево}{8_y2_-0_45}{lab8:y2:1}
\myImage{($\alpha = -0.44$) Второе решение разошлось, первое циклично двигается, заворачиваясь в петельку}{8_y2_-0_44}{lab8:y2:2}
\myImage{($\alpha = 0.44$) Аналогичные результаты, только петелька отразилась сверху вниз}{8_y2_0_44}{lab8:y2:3}
\clearpage
\myImage{($\alpha = 0.45$) Первое решение разошлось вправо}{8_y2_0_45}{lab8:y2:4}

Как можно заметить, симметрия относительно значений $\alpha$
наблюдается даже в этом эксперименте.


\clearpage
\section{Влияние случайного шума на вторую
переменную системы}

В данном эксперименте мы добавляем влияние шума
ко второй переменной системы, получая уравнение \eqref{app:rand:1}.
Начальные значения решений и метод аналогичны разделу \ref{lab9}.

\begin{equation}\label{app:rand:1}
\begin{cases}
    &\mathrm{d}y_1 = y_2\mathrm{d}t\\
    &\mathrm{d}y_2 = (-3y_2^3\ + \nu y_2 - y_1)\mathrm{d}t + \sigma * \mathrm{d}W
\end{cases}
\end{equation}

\myImage{($\alpha = 0.1$) Решения более зашумленные вдоль оси $Oy_2$}{9_y2_0_1}{lab9:y2:1}
\myImage{($\alpha = 3$) Решения не разошлись, напоминает НЛО}{9_y2_3_ok}{lab9:y2:2}
\myImage{($\alpha = 3$) В этой реализации случайное воздействие сильно повлияло на второе решение, и оно разошлось}{9_y2_3}{lab9:y2:3}


\end{document}

%%% Local Variables:
%%% mode: latex
%%% TeX-master: t
%%% End:
