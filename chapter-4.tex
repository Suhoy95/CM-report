\chapter{Определение устойчивости через мультипликаторы системы}\label{lab4}

В этом разделе будет проведено исследование нашего предельного цикла
на асимптотическую орбитальную устойчивость. Проверку будем проводить с помощью
аналога теоремы Андронова-Витта, вычислив мультипликаторы системы первого
приближения вдоль исследуемого предельного цикла.

Дадим необходимые для этого определения. Под \textit{орбитальной устойчивостью}
мы будем подразумевать такое свойство периодического решения системы $\phi(t)$,
если для любого $\epsilon > 0$ найдется постоянное число
$\delta=\delta(\epsilon) > 0$, такое, что траектория всякого решения $X(t)$,
начинающегося в $\delta$-окрестности траектории $\phi(t)$, остается в
$\epsilon$-окрестности траектории $\phi(t)$ при всех $t \geq 0$.

\textit{Асимптотическим} же будем считать такое орбитально устойчивое решение, что
выполняется условие \ref{lab4:asm_o_u}, то есть любая достаточно близкая
траектория сходится к орбитально устойчивому решению:

\begin{equation}\label{lab4:asm_o_u}
    \exists \alpha, \beta > 0:
    \left\| {\mathbf{X}\left( 0 \right) -
            \boldsymbol{\varphi} \left( 0 \right)}
    \right\| < \delta
    \rightarrow
    \left\| {\mathbf{X}\left( t \right) -
            \boldsymbol{\varphi} \left( t \right)}
    \right\| \le \alpha
    \left\| {\mathbf{X}\left( 0 \right) -
            \boldsymbol{\varphi} \left( 0 \right)}
    \right\|{e^{ - \beta t}}
\end{equation}

Из поставленных определений видна мотивация исследования решения на наличие
асимптотической орбитальной устойчивости: подтвердив его, мы, увеличивая
точность вычислений, можем быть уверены, что лучше приближаемся к необходимой
траектории.

В этом нам поможет, аналог \textit{теоремы Андронова-Витта}: Если имеется
периодическое решение автономной системы и его система первого приближения
имеет два мультипликатора, один равный единице, а второй по-модулю меньше
единици, то полученное периодическое решение асимптотически орбитально устойчиво.

\clearpage

Мультипликаторами называются значения величин, полученных в результате
алгоритма, изученного на лекционных занятиях:

\begin{itemize}
    \item Рассмотрим систему $\dot{x} = F(x)$ и периодическое решение $\eta(t)$;
    \item Выразим линеаризированную систему $\dot{y} = F'(\eta(t))y$ вдоль
    данного решения;
    \item Вычислим ее вдоль периодического решения с н.у. (1, 0) и (0, 1);
    \item Получим матрицу монодромии $\Phi = (\phi_1, \phi_2)$, где $\phi_1, \phi_2$ -
    решения системы полученные на предыдущем шаге.
    \item Собственные числа матрицы монодромии мы и будем считать мультипликаторами
    системы.
\end{itemize}

Осталось, провести поставленные шаги. Рассмотрим нашу систему \ref{lab4:eq:1}:
\begin{equation}\label{lab4:eq:1}
    \begin{cases}
        &\dot{y_1} = y_2 = f_1(y_1, y_2) \\
        &\dot{y_2} = -3y_2^3\ + \nu y_2 - y_1 = f_2(y_1, y_2)
    \end{cases}
\end{equation}

Посчитаем Якобиан нашей системы (уравнения \ref{lab4:eq:2}):

\begin{equation}\label{lab4:eq:2}
    \frac{\partial f_1}{\partial y_1} = 0;\qquad
    \frac{\partial f_1}{\partial y_2} = 1;\qquad
    \frac{\partial f_2}{\partial y_1} = -1;\qquad
    \frac{\partial f_2}{\partial y_2} = -9y_2^2 + \nu;
\end{equation}

И строим линеаризированную систему вдоль цикла $\eta(t)$:

\begin{equation}\label{lab4:eq:3}
    \begin{cases}
        &\dot{Y_1} = Y_2 \\
        &\dot{Y_2} = -Y_1 + (-9\eta_2^2(t) + \nu)Y_2
    \end{cases}
\end{equation}

В листинге \ref{lab4:prog:1} мы, методом Эйлера решаем полученную систему,
находим матрицу монодромии и вычисляем ее Собственные числа (Eigenvalues).
Как видно из вывода программы, вычисленные мультипликаторы удовлетворяют
условиям рассмотренной теоремы с искомой точностью до 3-х знаков
($8.59 * 10^{-4}$ и $1$), таким исследуемый цикл асимптотически орбитально устойчив.

\begin{program}
    \caption{Вычисление мультипликаторов}
    \label{lab4:prog:1}
    \begin{verbatim}
def linear_form(y1_0, y2_0, cycle_y1, cycle_y2):
    # Решение линеаризованной системы вдоль цикла
    y1 = [y1_0]
    y2 = [y2_0]
    for j in range(len(cycle_y1)):
        y1.append(y1[-1] + h * (y2[-1]))
        y2.append(y2[-1] + h * (-y1[-1] +
                       (-9*cycle_y2[j] ** 2 + ny) * y2[-1]))
    return [y1[-1], y2[-1]]

cycle_y1, cycle_y2 = line(0.724197, 0) # вычисление цикла
# решение линеаризированной системы с н. у. (1, 0) и (0, 1)
f1 = linear_form(1, 0, cycle_y1, cycle_y2)
f2 = linear_form(0, 1, cycle_y1, cycle_y2)
f = np.array([ # Матрица монодромии
    [f1[0], f2[0]],
    [f1[1], f2[1]],
])
print("F-matrix:")
print(f)
# Вычисление собственных чисел матрицы монодромии
p = np.linalg.eig(f)
print("Eigenvalues:")
print(p[0])
# Вывод программы:
# F-matrix:
# [[  8.92710796e-04   6.63982838e-05]
#  [  5.05095135e-01   1.00018677e+00]]
# Eigenvalues:
# [  8.59150781e-04   1.00022033e+00]
    \end{verbatim}
\end{program}