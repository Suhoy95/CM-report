\chapter{Определение устойчивости через мультипликаторы системы}\label{lab4}

В этом разделе будет проведено исследование нашего предельного цикла
на асимптотическую орбитальную устойчивость. Проверку будем проводить с помощью
аналога теоремы Андронова-Витта, вычислив мультипликаторы системы первого
приближения вдоль исследуемого предельного цикла.

Дадим необходимые для этого определения\cite{bookdiff}, относительно системы
дифференциальных уравнений \eqref{lab4:eq:du}.

\begin{equation}\label{lab4:eq:du}
    \dot{x} = f(x), \quad x \in R^n \quad f \in C[R^n]
\end{equation}

\begin{definition}
    Пусть $x = \eta(t)$ - решение системы \eqref{lab4:eq:du},
    определенное при $t > 0$. \textbf{Положительной полутраекторией}
    решения назовем множество в фазовом пространстве:
    \begin{equation*}
        L^+[\eta(\cdot)] = \{x\in R^n, \quad x = \eta(t), \quad t \geq 0\}.
    \end{equation*}
\end{definition}

\begin{definition}
    Решение $\eta(t)$ системы \eqref{lab4:eq:du} называется
    \textbf{орбитально устойчивым} при $t \rightarrow \infty$,
    если для любого $\epsilon > 0$ найдется $\delta > 0$
    такое, что для всех других решений $x(t)$
    системы \eqref{lab4:eq:du} c условием \\
    $||x(0) - \eta(0)|| < \delta$ выполняется
    $\rho(x(t), L^+[\eta(\cdot)]) < \epsilon$ для всех $t \geq 0$.
    Здесь $\rho(x,L)$ означает расстояние от точки $x$ до множества
    $L$ в пространстве $R^n$.
\end{definition}

\begin{definition}
    Орбитально устойчивое решение $\eta(t)$ называется
    \textbf{асимптотически орбитально устойчивым},
    если существует $\Delta > 0$ такое, что для всех решений
    $x(t)$, удовлетворяющих соотношению $||x(0) - \eta(0)|| < \Delta$,
    выполняется предельное соотношение $\rho(x(t), L^+[\eta(\cdot)]) \rightarrow 0$
    при $t \rightarrow \infty$.
\end{definition}

Из поставленных определений видна мотивация исследования решения на наличие
асимптотической орбитальной устойчивости: подтвердив его, мы, увеличивая
точность вычислений, будем уверены, что приближаемся к искомой
траектории. В этом нам поможет, аналог \textit{теоремы Андронова-Витта}.

\begin{theorem}
    \textbf{теорема Андронова-Витта}. Если имеется
    периодическое решение автономной системы и его система первого приближения
    имеет два мультипликатора, один равный единице, а второй по-модулю меньше
    единицы, то полученное периодическое решение асимптотически орбитально устойчиво.
\end{theorem}

Мультипликаторами называются значения величин, полученных в результате
алгоритма, изученного на лекционных занятиях:

\begin{itemize}
    \item Рассмотрим систему $\dot{x} = F(x)$ и периодическое решение $\eta(t)$;
    \item Выразим линеаризированную систему $\dot{y} = F'(\eta(t))y$ вдоль данного решения;
    \item Вычислим ее вдоль периодического решения с н.у. (1, 0) и (0, 1);
    \item Получим матрицу монодромии $\Phi = (\phi_1, \phi_2)$, где $\phi_1, \phi_2$ -
    решения системы, полученные на предыдущем шаге.
    \item Собственные числа матрицы монодромии и будут мультипликаторами системы.
\end{itemize}

Рассмотрим нашу систему \eqref{lab4:eq:1}:
\begin{equation}\label{lab4:eq:1}
    \begin{cases}
        &\dot{y_1} = y_2 = f_1(y_1, y_2) \\
        &\dot{y_2} = -3y_2^3\ + \nu y_2 - y_1 = f_2(y_1, y_2)
    \end{cases}
\end{equation}

Посчитаем Якобиан нашей системы (уравнения \eqref{lab4:eq:2}):

\begin{equation}\label{lab4:eq:2}
    \frac{\partial f_1}{\partial y_1} = 0;\qquad
    \frac{\partial f_1}{\partial y_2} = 1;\qquad
    \frac{\partial f_2}{\partial y_1} = -1;\qquad
    \frac{\partial f_2}{\partial y_2} = -9y_2^2 + \nu;
\end{equation}

И строим линеаризированную систему вдоль цикла $\eta(t)$:

\begin{equation}\label{lab4:eq:3}
    \begin{cases}
        &\dot{Y_1} = Y_2 \\
        &\dot{Y_2} = -Y_1 + (-9\eta_2^2(t) + \nu)Y_2
    \end{cases}
\end{equation}

В листинге \ref{lab4:prog:1} мы, методом Эйлера решаем вычисленную систему
и находим Собственные числа (Eigenvalues) матрицы монодромии.
Как видно из вывода программы, вычисленные мультипликаторы удовлетворяют
условиям теоремы с искомой точностью ($8.59 * 10^{-4}$ и $1$),
таким образом исследуемый цикл асимптотически орбитально устойчив.

\begin{program}
    \caption{Вычисление мультипликаторов}
    \label{lab4:prog:1}
    \begin{verbatim}
def linear_form(y1_0, y2_0, cycle_y1, cycle_y2):
    # Решение линеаризованной системы вдоль цикла
    y1 = [y1_0]
    y2 = [y2_0]
    for i in range(len(cycle_y1)):
        y1.append(y1[i] + h * (y2[i]))
        y2.append(y2[i] + h * (-y1[i] +
                       (-9*cycle_y2[i] ** 2 + ny) * y2[i]))
    return [y1[-1], y2[-1]]

cycle_y1, cycle_y2 = line(0.72424, 0) # вычисление цикла
# решение линеаризированной системы с н. у. (1, 0) и (0, 1)
f1 = linear_form(1, 0, cycle_y1, cycle_y2)
f2 = linear_form(0, 1, cycle_y1, cycle_y2)
f = np.array([ # Матрица монодромии
    [f1[0], f2[0]],
    [f1[1], f2[1]],
])
print("F-matrix:")
print(f)
# Вычисление собственных чисел матрицы монодромии
p = np.linalg.eig(f)
print("Eigenvalues:")
print(p[0])
# Вывод программы:
# F-matrix:
# [[  8.56578243e-04  -3.73826069e-06]
#  [  5.05107751e-01   1.00012266e+00]]
# Eigenvalues:
# [  8.58467858e-04   1.00012077e+00]
    \end{verbatim}
\end{program}