\chapter{Влияние постоянного запаздывания на систему}

С этой главы мы познакомимся с \textit{функциональными дифференциальными
уравнениями (ФДУ)}, а точнее с системами в которых присутствует элемент
запаздывания и как он влияет на поведение исследуемой системы.

\begin{definition}
    Функциональным дифференциальным уравнением с постоянным запаздыванием,
    мы будем называть уравнение вида \ref{lab6:eq1}.
\end{definition}

\begin{equation}\label{lab6:eq1}
    \begin{cases}
        &\dot{x} = f(t, x, x_t(t - \tau)) \\
        &x(0) = x_0 \\
        &x_{t_0}(\cdot) = \begin{Bmatrix}y_0(t); t \in [\tau;0)\end{Bmatrix}
    \end{cases}
\end{equation}

Посмотрев на него, можно выделить основные особенности от обычных дифференциальных
уравнений, помимо зависимости от $t$ и $x$, у нас появляется зависимость от
значения фазовой переменной в момент $x_t(t-\tau)$ (Здесь и далее, $x_t$ - \textit{функция
предыстории}, описывающая значения фазовой переменной на интервале $[t-\tau;t)$).
Так же, чтобы получить однозначное решение системы \ref{lab6:eq1}, понадобилось
задать предысторию до момента $t_0 = 0$ (третье равенство системы).

\section{Добавление запаздывания по $y_1$ в систему}

Проведем первый эксперимент с постоянным запаздыванием.
Для этого, введем в нашу систему слагаемое $\alpha * y_1(t-\tau)$ (Ур. \ref{lab6:eq2})
При параметре $\alpha = 0$ оно не влияет на систему, следовательно мы будем
наблюдать наш предельный цикл. Постепенно изменяя этот параметр мы будем наблюдать
на изменения характера системы.

\begin{equation}\label{lab6:eq2}
\begin{cases}
    &\dot{y_1} = y_2 + \alpha * y_1(t-\tau)\\
    &\dot{y_2} = -3y_2^3\ + \nu y_2 - y_1
\end{cases}
\end{equation}

В программе \ref{lab6:prog:1} показано основное изменения нашего
эксперимента. Теперь во входные параметры мы уже передаем не
одну точку, а массив размера len(y1\_0), и при вычислении следующего
шага в методе Эйлера, мы обращаемся к (i - len(y1\_0) + 1)-му элементу массива.

В эксперименте рассматриваются два начальных условия: система находилась
в точке (0.1, 0.1) в 100 первых итерациях, и в точке (2, 2) в 100 последних итерациях.

Также стоит отметить, модернизацию нашей программы в техническом виде,
с помощью слайдера оказалось неудобным отлаживать подбор необходимого параметра.
Поэтому, чтобы упростить эксперименты и не перезапускать программу, вместо
слайдера был поставлен \textmd{TextBox} - поле для ввода текста, куда мы
можем ввести любое, необходимое нам, значение (пересчет графика произойдет при нажатии на клавишу
\textmd{Enter}). При желании, с подробностями использования этого виджета можно ознакомиться
в приложении.

\begin{program}
    \caption{Программирование постоянной задержки}
    \label{lab6:prog:1}
    \begin{verbatim}
def line(y1_0, y2_0, alpha):
    # решение задачи методом Эйлера
    y1 = copy.deepcopy(y1_0)
    y2 = copy.deepcopy(y2_0)
    h = 0.03
    for i in range(len(y1_0)-1, 20000):
        # Запаздывание влияет на y1
        y1.append(y1[i] + h * (y2[i] +
                             alpha * y1[i - len(y1_0) + 1]))
        y2.append(y2[i] +
                   h * (-3*y2[i] ** 3 + ny * y2[i] - y1[i]))
    ax.plot(y1, y2)

# ...
# Расчет примеров
line([0.1 for a in range(100)],
     [0.1 for a in range(100)],
     alpha)
line([2 for a in range(100)],
     [2 for a in range(100)],
     alpha)
    \end{verbatim}
\end{program}

\clearpage

Сразу стоит отметить, что из-за внесения задержки, система начинала приобретать
непредсказуемый характер. В некоторых случаях это вызывало переполнение переменных,
что мешало пронаблюдать эксперимент.

\myImage{($\alpha = -0.65$) Решение от точки (0.1, 0.1) хаотично расходится от центра.
Решение от точки (2, 2) не смогло построится}{6_y1_-0_65}{lab6:y1:1}
\myImage{($\alpha = -0.649$) Оба решения построились. Отчетливо наблюдается жесткость
системы, цикл сохранился, но об изолированности говорить не приходится}{6_y1_-0_649}{lab6:y1:2}
\myImage{($\alpha = 0.426$) Решение все еще сходится к циклу, но постепенно разбалтывается}{6_y1_0_426}{lab6:y1:3}
\myImage{($\alpha = 0.427$) Второе решение перестало моелироваться. Первое все еще сходится к циклу}{6_y1_0_427}{lab6:y1:4}
\myImage{($\alpha = 1$) У первого решения появляются две петельки и траетория становится более хаотична}{6_y1_1_0}{lab6:y1:5}
\myImage{($\alpha = 1.24$) Первое решение хаотично изменятся в пределах цикла}{6_y1_1_14}{lab6:y1:6}


\clearpage
\section{Добавление запаздывания по $y_2$ в систему}

\myImage{}{6_y2_-5_1}{lab6:y2:1}
\myImage{}{6_y2_0_5}{lab6:y2:2}
\myImage{}{6_y2_0_99}{lab6:y2:3}
\myImage{}{6_y2_1_0}{lab6:y2:4}
\myImage{}{6_y2_1_02}{lab6:y2:5}
\myImage{}{6_y2_1_025}{lab6:y2:6}
