% Добрый день, Илья, высылаю список замечаний по оформлению.
% 1. [исправлено] Стр 2. В аннотации нужно убрать слово "Отчет", и написать "Аннотация" или вообще не писать заголовок.
% 2. [исправлено, на \eqref, стили поправлены] Стр 4 и далее. Ссылки на формулы должны быть (1.2), а не 1.2. ТЕХ это делает автоматически, но вы, по-видимому, делаете вручную или делаете неправильно.
% 3. [исправлено] Стр 6 и далее. Слово "График" в подписях к рисункам надо заменить на "Рис.". Так принято.
% 4. [исправлено] Стр. 9. Ошибка: вместо "система 1.2" должно быть "система (1.3)".
% 5. [исправлено, взято из "глав ДУ"] Стр 14. Определение 3 не расписано в формулах и его можно понять двояко. Определение 4 неверно.
% 6. Стр 19 и далее. Запись ФДУ неверна. Вы пытаетесь одновременно написать и запаздывание и функцию предысторию.
% 7. Стр. 25. Непонятна связь между Т и тау в примере.
% 8. Стр. 30. Опечатка: "вкруг".
% 9. Стр. 34. Предложения, начинающиеся со слов "Где", "Тут", "Которых" неверно построены стилистически.
% 10. Стр 38. Последнее предложение нужно убрать из заключения по разным причинам.
% 11. Стр. 39. Список литературы оформлен неверно, не по стандартам. Также отсутствует ссылка на литературу,
% содержащую стохастические дифференциальные уравнения, например,
%    Кузнецов Д.Ф. Численное моделирование стохастических дифференциальных уравнений и стохастических интегралов. С.Петербург. Наука. 1999.


\begin{abstract}

Работа описывает исследование параметризованной нелинейной
системы на наличие предельных циклов. В ходе исследования
системы, рассматриваются следующие вопросы: нахождение
параметра системы, при котором наблюдается предельный цикл;
поиск параметра, где наблюдается бифуркация поведения
системы; исследование свойств обнаруженного предельного цикла;
определение характера его устойчивости; исследование влияния
разных видов запаздывания на систему (постоянное, переменное
и распределенное запаздывание), а также влияния
стохастического воздействия на систему. Данные задачи
изучаются путем проведения численных экспериментов
с помощью интерпретатора Python 3.5 и математических
библиотек (numpy\cite{numpy}, matplotlib\cite{matplotlib}).

\end{abstract}

%%% Local Variables:
%%% mode: latex
%%% TeX-master: "rpz"
%%% End:
