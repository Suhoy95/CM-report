\chapter{Дополнительные эксперименты}

В данном приложении собраны численные эксперименты, которые
не вошли в основное содержание отчета, чтобы не загромождать
изложение материала, но тем не менее не стали менее важными
в контексте проведенных лабораторных работ.

\section{Влияние постоянного запаздывания на вторую
переменную системы}

В данном эксперименте мы добовляем постоянное запаздывание
ко второй переменной системы, получая уравнение \ref{lab6:eq3}.
Начальные значения решений и метод аналогичны разделу \ref{lab6}.

\begin{equation}\label{lab6:eq3}
    \begin{cases}
        &\dot{y_1} = y_2\\
        &\dot{y_2} = -3y_2^3\ + \nu y_2 - y_1  + \alpha * y_1(t-\tau)
    \end{cases}
\end{equation}

\myImage{($\alpha = -5.1$) При отрицательном параметре изменяется форма цикла. Можно предполагать,
что предельность данного цикла остается}{6_y2_-5_1}{lab6:y2:1}
\myImage{($\alpha = 0.5$) Цикл начинает раскручиваться в правую сторону}{6_y2_0_5}{lab6:y2:2}
\myImage{($\alpha = 0.99$) Наблюдается очень плотная спираль}{6_y2_0_99}{lab6:y2:3}
\myImage{($\alpha = 1$) Потенциальная точка бифуркации. Два решения сошлись к двум разным циклам}{6_y2_1_0}{lab6:y2:4}
\myImage{($\alpha = 1.02$) Даже небольшое увеличение $\alpha$ изменило систему}{6_y2_1_02}{lab6:y2:5}
\clearpage
\myImage{($\alpha = 1.025$) Решение начинает увеличиваться по $y_1$}{6_y2_1_025}{lab6:y2:6}
\myImage{($\alpha = 1.13$) В интервале от 600 до 1000 наблюдаются резкие колебания}{6_y2_1_13}{lab6:y2:7}
Дальнейшее увеличение $\alpha$ приводили к резкому увеличению значения координат,
из-за чего вызывались ошибки в моделирующей системе.