\chapter{Дополнительные эксперименты}\label{app}

В данном приложении собраны численные эксперименты, которые
не вошли в основное содержание отчета, чтобы не загромождать
изложение материала, но тем не менее не стали менее важными
в контексте проведенных лабораторных работ.

\section{Влияние постоянного запаздывания на вторую
переменную системы}\label{app:const}

В данном эксперименте мы добавляем постоянное запаздывание
ко второй переменной системы, получая уравнение \eqref{app:const:1}.
Начальные значения решений и метод аналогичны разделу \ref{lab6}.

\begin{equation}\label{app:const:1}
    \begin{cases}
        &\dot{y_1} = y_2\\
        &\dot{y_2} = -3y_2^3\ + \nu y_2 - y_1  + \alpha * y_1(t-\tau)
    \end{cases}
\end{equation}

\myImage{($\alpha = -5.1$) При отрицательном параметре изменяется форма цикла. Можно предполагать,
что предельность данного цикла остается}{6_y2_-5_1}{lab6:y2:1}
\myImage{($\alpha = 0.5$) Цикл начинает раскручиваться в правую сторону}{6_y2_0_5}{lab6:y2:2}
\myImage{($\alpha = 0.99$) Наблюдается очень плотная спираль}{6_y2_0_99}{lab6:y2:3}
\myImage{($\alpha = 1$) Потенциальная точка бифуркации. Два решения сошлись к двум разным циклам}{6_y2_1_0}{lab6:y2:4}
\myImage{($\alpha = 1.02$) Даже небольшое увеличение $\alpha$ изменило систему}{6_y2_1_02}{lab6:y2:5}
\clearpage
\myImage{($\alpha = 1.025$) Решение начинает увеличиваться по $y_1$}{6_y2_1_025}{lab6:y2:6}
\myImage{($\alpha = 1.13$) В интервале от 600 до 1000 наблюдаются резкие колебания}{6_y2_1_13}{lab6:y2:7}

\clearpage
\section{Влияние переменного запаздывания на вторую
переменную системы}\label{app:changeable}

В данном эксперименте мы добавляем переменное запаздывание
ко второй переменной системы, получая уравнение \eqref{app:change:1}.
Начальные значения решений и метод аналогичны разделу \ref{lab7}.

\begin{equation}\label{app:change:1}
    \begin{cases}
        &\dot{y_1} = y_2\\
        &\dot{y_2} = -3y_2^3\ + \nu y_2 - y_1 + \alpha * y_1(t-\tau*\sin(\frac{2t}{T}))
    \end{cases}
\end{equation}

\myImage{($\alpha = -21$) В цикле начались резкие колебания}{7_y2_-21}{lab7:y2:1}
\myImage{($\alpha = -11$) Зарождения колебаний в цикле }{7_y2_-11}{lab7:y2:2}
\myImage{($\alpha = 0.7$) Цикл меняет свою форму и напоминает виниловую пластинку}{7_y2_0_7}{lab7:y2:3}
\myImage{($\alpha = 0.99$) Цикл начинает смещаться вправо }{7_y2_0_99}{lab7:y2:4}
\myImage{($\alpha = 1$) Жесткая бифуркация: каждое решение сошлось к своему циклу}{7_y2_1}{lab7:y2:5}
\myImage{($\alpha = 1.1$) Оба решения начали уходить в бесконечность вдоль асимптоты}{7_y2_1_1}{lab7:y2:6}

\clearpage
\section{Влияние распределенного запаздывание на вторую
переменную системы}

В данном эксперименте мы добавляем распреденое запаздывание
ко второй переменной системы, получая уравнение \eqref{app:raspr:1}.
Начальные значения решений и метод аналогичны разделу \ref{lab8}.

\begin{equation}\label{app:raspr:1}
  \begin{cases}
      &\dot{y_1} = y_2\\
      &\dot{y_2} = -3y_2^3\ + \nu y_2 - y_1 + \alpha * \int_{t-\tau}^t y_1^2(s)\mathrm{d}s
  \end{cases}
\end{equation}

\myImage{($\alpha = -0.45$) Хаотично покрутившись в районе цикла, первое решение разошлось влево}{8_y2_-0_45}{lab8:y2:1}
\myImage{($\alpha = -0.44$) Второе решение разошлось, первое циклично двигается, заворачиваясь в петельку}{8_y2_-0_44}{lab8:y2:2}
\myImage{($\alpha = 0.44$) Аналогичные результаты, только петелька отразилась сверху вниз}{8_y2_0_44}{lab8:y2:3}
\clearpage
\myImage{($\alpha = 0.45$) Первое решение разошлось вправо}{8_y2_0_45}{lab8:y2:4}

Как можно заметить, симметрия относительно значений $\alpha$
наблюдается даже в этом эксперименте.


\clearpage
\section{Влияние случайного шума на вторую
переменную системы}

В данном эксперименте мы добавляем влияние шума
ко второй переменной системы, получая уравнение \eqref{app:rand:1}.
Начальные значения решений и метод аналогичны разделу \ref{lab9}.

\begin{equation}\label{app:rand:1}
\begin{cases}
    &\mathrm{d}y_1 = y_2\mathrm{d}t\\
    &\mathrm{d}y_2 = (-3y_2^3\ + \nu y_2 - y_1)\mathrm{d}t + \sigma * \mathrm{d}W
\end{cases}
\end{equation}

\myImage{($\alpha = 0.1$) Решения более зашумленные вдоль оси $Oy_2$}{9_y2_0_1}{lab9:y2:1}
\myImage{($\alpha = 3$) Решения не разошлись, напоминает НЛО}{9_y2_3_ok}{lab9:y2:2}
\myImage{($\alpha = 3$) В этой реализации случайное воздействие сильно повлияло на второе решение, и оно разошлось}{9_y2_3}{lab9:y2:3}
