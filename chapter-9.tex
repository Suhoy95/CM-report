\chapter{Влияние случайного шума}\label{lab9}
Последним видом моделируемых систем, с которым мы познакомимся
в этой работе, будет моделирование \textit{стохастического
дифференциального уравнения}\cite{sdu}. Оно описывает систему, на работу
которой в каждый момент времени может оказывать влиять
случайное искажение (шум). Обычно, в качестве шума рассматривают
реализацию выборки стандартного нормального распределения,
поэтому в наших экспериментах мы будем использовать его.

В том виде, в котором описывались предыдущие дифференциальные
уравнения, мы описывать данное уравнение не можем, однако
такое уравнение хорошо описывается через интегральную форму
\eqref{lab9:eq:integral}.

\begin{equation}\label{lab9:eq:integral}
    x(t) = x_0 + \int_{t_0}^t f(s,x(s))\mathrm{d}s +
                 \int_{t_0}^t \sigma(W, x(W))\mathrm{d}W
\end{equation}

Второе слагаемое уравнения \eqref{lab9:eq:integral} является
интегралом по Винеровскому процессу, который как раз и
описывает приращение ошибки (шума) в системе. От такого вида
уравнения можно перейти к СДУ в виде дифференциалов
\eqref{lab9:eq:dif}.

\begin{equation}\label{lab9:eq:dif}
    \mathrm{d}x = f(t, x)\mathrm{d}t +
                  \sigma(t, x(t))\mathrm{d}W
\end{equation}

Для нашей системы, такое уравнение примет вид \eqref{lab9:eq:our}.

\begin{equation}\label{lab9:eq:our}
\begin{cases}
    &\mathrm{d}y_1 = y_2\mathrm{d}t + \sigma * \mathrm{d}W\\
    &\mathrm{d}y_2 = (-3y_2^3\ + \nu y_2 - y_1)\mathrm{d}t
\end{cases}
\end{equation}

В этом уравнении мы добавляем шум к изменению первой координаты
системы, Виноровский процесс описывается через стандартное
нормальное распределение, а за счет коэффициента $\sigma$ мы
сожем влиять на среднеквадратичное отклонение данного процесса,
а следовательно, и на силу влияния данного процесса на систему.

Для моделирования такой системы используется другая
модификация метода Эйлера - \textit{метод Эйлера-Марайамы}.
Для нашей системы данный метод будет выглядеть так:

\begin{equation}\label{lab9:eq:method}
\begin{cases}
    &y_1^{i+1} = y_1^i + h y_2^i + \sigma * \sqrt{h} W^i\\
    &y_2^{i+1} = y_2^i + h (-3(y_2^i)^3\ + \nu y_2^i - y_1^i)
\end{cases}
\end{equation}

$W^i$ - реализация генератора случайных чисел стандартного
нормального распределения. Начальные значения в текущем
эксперименте не принципиальны. Поэтому быле взяты предыдущие
точки $(0.1,0.1)$ и $(2,2)$, к которым добавилось.
случайное воздействие $\sigma W^0$.

\begin{definition}
    Метод моделирующий стохастическое уравнение сходится с
    порядком $p$ в сильном смысле, если
    $\exists C : M(|X(T) - x^N|) \leq Ch^p$, где $X(t)$ - настоящее решение
    данного уравнения, а ${x}_i$ - решение, полученное с помощью метода.
\end{definition}

\begin{definition}
    Метод моделирующий стохастическое уравнение сходится с
    порядком $p$ слабо, если
    $\exists C : M(\int_{t_0}^T(X(t) - x^i)^2dt)^(\frac{1}{2}) \leq Ch^p$
\end{definition}

\begin{theorem}
    Метод Эйлера-Марайамы сходится, как в сильном, так и в слабом
    смысле с порядком $p = \frac{1}{2}$.
\end{theorem}

\myImage{($\sigma = 0.1$) видно, что цикл сохранил форму, но контур стал размываться вдоль оси $Oy_1$}{9_y1_0_1}{lab9:y1:1}
\myImage{($\sigma = 3$) Цикл напоминает размытое пятно, но значения решения не рассходятся}{9_y1_3_ok}{lab9:y1:2}
\myImage{($\sigma = 3$) Значение $\sigma$ не поменялось, но при определенной выборке решение разошлось}{9_y1_3}{lab9:y1:3}

\clearpage
Таким образом видно, что при добавлении шума в небольшом количестве,
поведение системы остается в пределах известного поведения, хоть
и становится слишком размытым. При сильном увеличении влияния
процесс может, как разойтись, так и нет, в зависимости от
реализации выборки случайной величины.
